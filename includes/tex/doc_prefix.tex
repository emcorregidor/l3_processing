\title{L3 Acquisition: Does it Follow Language Order or Language Proximity?}
\author{Eva María Corregidor Luna}
\organization{Rutgers University}
\email{eva.corregidor@rutgers.edu}


\maketitle

\begin{abstract}
There is still a lot we do not know about L3 acquisition of sounds. Assuming what the L2LP model (van Leussen and Escudero, 2015) teaches, L3 learners will fully transfer the perceptual space of one of their languages and shape it to the new L3. While different factors could contribute to the choice of transferring the L1 or L2 percepcutal space, there are two which seem particularly central: language proximity and language order. This study aims to learn what factor does the L3 learner consider more influential. Thirty advanced bilinguals (15 L1 English, L2 Spanish; 15 L1 Spanish, L2 English) who were early learners of Korean completed a discrimination task and a reading aloud task so to learn what perceptual space were they fully transfering when restructuring it into their new language.
\end{abstract}

\keywords{L3 acquisition, phonetics, vowels, language proximity, language order}

