% Options for packages loaded elsewhere
\PassOptionsToPackage{unicode}{hyperref}
\PassOptionsToPackage{hyphens}{url}
%
\documentclass[
  a4paper,
  11pt,
  twocolumn]{article}
\usepackage{amsmath,amssymb}
\usepackage{iftex}
\ifPDFTeX
  \usepackage[T1]{fontenc}
  \usepackage[utf8]{inputenc}
  \usepackage{textcomp} % provide euro and other symbols
\else % if luatex or xetex
  \usepackage{unicode-math} % this also loads fontspec
  \defaultfontfeatures{Scale=MatchLowercase}
  \defaultfontfeatures[\rmfamily]{Ligatures=TeX,Scale=1}
\fi
\usepackage{lmodern}
\ifPDFTeX\else
  % xetex/luatex font selection
\fi
% Use upquote if available, for straight quotes in verbatim environments
\IfFileExists{upquote.sty}{\usepackage{upquote}}{}
\IfFileExists{microtype.sty}{% use microtype if available
  \usepackage[]{microtype}
  \UseMicrotypeSet[protrusion]{basicmath} % disable protrusion for tt fonts
}{}
\makeatletter
\@ifundefined{KOMAClassName}{% if non-KOMA class
  \IfFileExists{parskip.sty}{%
    \usepackage{parskip}
  }{% else
    \setlength{\parindent}{0pt}
    \setlength{\parskip}{6pt plus 2pt minus 1pt}}
}{% if KOMA class
  \KOMAoptions{parskip=half}}
\makeatother
\usepackage{xcolor}
\usepackage{graphicx}
\makeatletter
\newsavebox\pandoc@box
\newcommand*\pandocbounded[1]{% scales image to fit in text height/width
  \sbox\pandoc@box{#1}%
  \Gscale@div\@tempa{\textheight}{\dimexpr\ht\pandoc@box+\dp\pandoc@box\relax}%
  \Gscale@div\@tempb{\linewidth}{\wd\pandoc@box}%
  \ifdim\@tempb\p@<\@tempa\p@\let\@tempa\@tempb\fi% select the smaller of both
  \ifdim\@tempa\p@<\p@\scalebox{\@tempa}{\usebox\pandoc@box}%
  \else\usebox{\pandoc@box}%
  \fi%
}
% Set default figure placement to htbp
\def\fps@figure{htbp}
\makeatother
\setlength{\emergencystretch}{3em} % prevent overfull lines
\providecommand{\tightlist}{%
  \setlength{\itemsep}{0pt}\setlength{\parskip}{0pt}}
\setcounter{secnumdepth}{5}
%------------------------------------------------------------------------------%
% PAPER TEMPLATE FOR ICPHS 2023 Prague                                         %
%                                                                              %
% Original template downloaded from:                                           %
% http://www.icphs2023.org/call-for-papers/                                    %
%                                                                              %
% Reformatted to work with Rmarkdown and R by:                                 %
% Joseph V. Casillas | Rutgers Univesity |11/11/2022                           %
%                                                                              %
% Available for download at:                                                   %
% https://github.com/jvcasillas/icphs2023_rmd_template                         %
%------------------------------------------------------------------------------%



% Packages
\usepackage{./includes/tex/icphs2023}
\usepackage{metalogo} 
\usepackage{epstopdf}
\usepackage{tipa}

% Links and urls must be black
%\hypersetup{urlcolor=black, citecolor=black, linkcolor=black}


% Packages removed from icphs2023.sty because of conflicts
% They have been added to the .Rmd yaml front matter
% \usepackage[latin1]{inputenc}
% \usepackage[T1]{fontenc}
% \usepackage[leqno,fleqn]{amsmath}
\usepackage[utf8]{inputenc}
\usepackage[T1]{fontenc}
\usepackage{bookmark}
\IfFileExists{xurl.sty}{\usepackage{xurl}}{} % add URL line breaks if available
\urlstyle{same}
\hypersetup{
  hidelinks,
  pdfcreator={LaTeX via pandoc}}

\author{}
\date{\vspace{-2.5em}}

\begin{document}

\title{L3 Acquisition: Does it Follow Language Order or Language Proximity?}
\author{Eva María Corregidor Luna}
\organization{Rutgers University}
\email{eva.corregidor@rutgers.edu}


\maketitle

\begin{abstract}
There is still a lot we do not know about L3 acquisition of sounds. Assuming what the L2LP model (van Leussen and Escudero, 2015) teaches, L3 learners will fully transfer the perceptual space of one of their languages and shape it to the new L3. While different factors could contribute to the choice of transferring the L1 or L2 percepcutal space, there are two which seem particularly central: language proximity and language order. This study aims to learn what factor does the L3 learner consider more influential. Thirty advanced bilinguals (15 L1 English, L2 Spanish; 15 L1 Spanish, L2 English) who were early learners of Korean completed a discrimination task and a reading aloud task so to learn what perceptual space were they fully transfering when restructuring it into their new language.
\end{abstract}

\keywords{L3 acquisition, phonetics, vowels, language proximity, language order}


\section{Introduction}

Speech is one of the unique features of human beings. In fact, before
the 28th week of gestation, human babies hearing system is already
operational \cite{Eggermont/Moore:2012}. As soon as we are born, we
count with the ability to discriminate phonetic contrasts of all
languages: universal speech perception \cite{Gervain:2022}. Through
sensory language and statistical learning, among others, they acquire
language-specific speech perception by the age of six months, and are
able to produce language-specific speech sounds by the age of ten months
\cite{Gervain:2022}.

However, there is a countless amount of factors that modify the
canonical timeline of acquisition and production of sounds. In the case
of bilinguals, early learners of two languages need to detect
language-specific patterns so to properly produces the sounds of each
language. On the other side, late bilinguals, sometimes described as
those who learn their second language by the age of 7, the acquisition
of sounds seems to work differently. Here, we will follow what the
Second Language Linguistic Perception Model (L2LP) assumes: late
learners fully copy their L1 perceptual system and adjust it when
necessary so to fit the L2 sounds \cite{vanLeussen/Escudero:2015}.

Similarly, the picture can get more complex when diving into the
polyglots context. In 2022, one-quarter (24.7\%) of working-age adults
(defined here as people aged 25-64 years) in the EU reported that they
knew 2 foreign languages, but also 12.3\% stated knowing 3 or more
\cite{Eurostat:2024}. Regardless of these numbers, the theoretical
background of L3 acquisition of sounds is still behind
\cite{Wang/Nance:2023}. This study aims to expand the L2LP to L3
learners by analyzing two of the most prominent factors that seem to
favor the fully transfer of the L1/L2 perceptual systems into the new L3
system so to modify it with the new L3 categories. The factors this
paper considers are language proximity and language acquisition order.
Lastly, given that most of the L3 research includes only Indo-European
languages \cite{Wang/Nance:2023}, this study includes a Koreanic
language: Korean.

\section{Literature Review}

\subsection{L2 Acquisition of Sounds}

L2 acquisition of sounds is a widely researched field, including
multiple models that diverse from each other in both their conclusions.
In this study, the model adopted is the L2LP
\cite{vanLeussen/Escudero:2015}), which states that the acquisition of
sounds is perceptually driven, and deals with the contrasts of the
sounds in the languages. The L2LP assumes that L2 learners create a L2
perceptual system by fully copying the L1 system. During the initial
stages of learning the L2 phonological system, the learners can
encounter three learning scenarios:

\begin{itemize}
\item Similar Scenario: the L2 categories match those of the L1, and the learner only needs to briefly adjust the boundaries of the sounds. This is seen as a relatively easy scenario for the learner.
\item New Scenario: the L2 categories do not match those of the L1, and the learner needs to form new phonetic categories. This is seen as a hard scenario for the learner.
\item Substet Scenario: the L2 categories are mapped to multiple L1 categories. This is seen as a hard scenario for the learner.
\end{itemize}

As a result, the L2LP assumes that the phonological systems of both
languages are separated and only activated when selected. The learners
moves from the initial to the developmental and to the end state through
L2 learning experience \cite{Wang/Nance:2023}.

\subsection{L3 Acquisition of Sounds}

Although the L2LP does not predict how do the L3 learners develop a
third perceptual system, or if they do so, researchers have adjusted
their assumptions and investigated L3 acquisition of sounds adapting
them \cite{Wang/Nance:2023}. Regardless of not having a great amount of
models to understand this phenomena, what remains clear is that L3
acquisition is more complex than L2 acquisition, for the significant
factors considered when talking about bilinguals are more tangled in the
case of multilinguals. Wang and Nance \cite{Wang/Nance:2023} reviewed
the L3 phonological acquisition experimental and theoretical studies
published up to date, identifying the following factors that contribute
to the transfer from the L1 and/or L2 into the L3 in terms of perception
and production:

\begin{itemize}
\item Language proximity
\item Proficiency
\item L2 status
\item L3 experience
\end{itemize}

In the case of language proximity, here it is understood as the degree
up to which two or more languages are similar in terms of typology,
sounds and grammar, among others. For instance, two languages belonging
to the same linguistic family might be closer, such as Spanish and
French, both being Romance languages. However, Spanish and English might
be less close, for English is a Germanic language. Ringbom and Jarvis
\cite{Ringbom/Jarvis:2009} indicate that L3 learners tend to elicit more
cross-linguistic transfer between languages when those are more similar.
For instance, Liu et al.~\cite{Liu/Zeng/Lu:2019} studied the acquisition
of voiced/voiceless stops in CH-EN bilinguals who were learning
different L3s (Japanese, Russian, and Spanish). In a discrimination
task, they found that the difficulty in perceiving L3 voiceless stops
was related to the similarity in the phonemic range of the learner's L1
and L2.

Regarding proficiency, research needs to specify if it is L1, L2 or L3
proficiency that is taken into account. Here, the level of proficiency
considered is that of the three languages. Others have already focused
on reviewing the impact of the proficiency in L3 learning. Cal and
Sypianska \cite{Cal/Sypianska:2020} measured the interaction between the
level of proficiency of the L2 and L3 in PO-EN bilinguals learning
Spanish. In a word reading task, the participants produced all Spanish
vowels /a, e, i, o, u/. The results indicated that the role of the L2
and L3 proficiency interacts with how each vowel is produced.

\subsection{Vowel Sounds}

While language proximity and proficiency are the factors analyzed in
this study, it is important to remark what linguistic trait the article
reviews. In this case, we have chosen vowel sounds. Language acquisition
of vowels is relevant for infants, who extract and generalize
repetition-based structures through the vocalic tier
\cite{Hochmann/etal:2011}, but also for those who need to learn foreign
vowels, more specifically distinguishing the contrast between their L1
and the learning languages.

When considering the common classification of vowels (tongue height,
backness and lip rounding; \cite{Hualde/Colina:2014}), along with the
number of vowel sounds, there are significant differences between
Spanish, English and Korean. While Spanish is a small system with 5
vowel sounds, English counts on a dense vowel space (12 vowel sounds).
Korean, however, sits somewhere in the middle, closer to English in
terms of numbers, for it is usually described as having 8 vowel sounds.

More specifically, all vowel sounds overlap across Spanish, English, and
Korean. For instance, the sounds /e/ and /u/, as shown in tables 1 and
2. All of them are relatively high-frequent, although /e/ might show
slight differences, for it tends to appear as the diphthong
\textipa{/eI/}. On the contrary, English and Korean share some sounds
that Spanish does not.

\begin{table}[!ht]
  \begin{center}
  \begin{tabular}{|c|c|c|c|}
  \hline
  \rowcolor[gray]{.75}
  Language & Word & Meaning (English) & IPA \\
  \hline
  Spanish & mesa & table & \textipa{/mesa/} \\
  English & bait & bait & \textipa{/beIt/} \\
  Korean  & bae  & pear/boat & \textipa{/pe/} \\
  \hline
  \end{tabular}
  \caption{Examples of words containing the vowel /e/ in Spanish, English, and Korean (romanized).}\label{tab:word_examples}
  \end{center}
\end{table}

\begin{table}[!ht]
  \begin{center}
  \begin{tabular}{|c|c|c|c|}
  \hline
  \rowcolor[gray]{.75}
  Language & Word & Meaning (English) & IPA \\
  \hline
  Spanish & luna & moon & \textipa{/luna/}\\
  English & boot & boot & \textipa{/but/} \\
  Korean  & mu   & radish & \textipa{/mu/} \\
  \hline
  \end{tabular}
  \caption{Examples of words containing the vowel /u/ in Spanish, English, and Korean (romanized).}\label{tab:word_examples_u}
  \end{center}
\end{table}

The sounds \textipa{/E/} and \textipa{/V/} are absent in Spanish but
present in English and Korean. This distribution can be seen in table 3.
As such, these four vowels can be used to learn about the L3 acquisition
factors that influences the choice of what language to transfer the
sounds from, an L1 or L2. In this case, Spanish-English bilinguals
learners of Korean would need to create a new sound category for the
Korean vowel if they are fully coping the Spanish system, but it would
not be needed if they were to copy the English system. Here, the factors
considered earlier, such as language proximity and proficiency might
shape this process.

\begin{table}[!ht]
  \begin{center}
  \begin{tabular}{|c|c|c|c|}
  \hline
  \rowcolor[gray]{.75}
  Vowel & Language & Word & IPA \\
  \hline
  \textipa{/E/} & Spanish & — & — \\
  \textipa{/E/} & English & bet & \textipa{/bEt/} \\
  \textipa{/E/} & Korean  & sae (bird) & \textipa{/sE/} \\
  \hline
  \textipa{/V/} & Spanish & — & — \\
  \textipa{/V/} & English & cut & \textipa{/kVt/} \\
  \textipa{/V/} & Korean  & eo (interjection “uh”) & \textipa{/V/} \\
  \hline
  \end{tabular}
  \caption{Examples of the vowels \textipa{/E/} and \textipa{/V/} in Spanish, English, and Korean (romanized).}\label{tab:vowel_examples}
  \end{center}
\end{table}

\section{This Study}

This study reviews L3 acquisition of sounds and asks what perception
system does the L3 learner fully copy, that of the L1 or the L2, and
what factor plays a stronger role in this transfer. For this, the
research question is (RQ1) what weights more in L3 sounds acquisition:
language order or language proximity? It is hypothesized that (H1)
bilinguals learning a new L3 fully transfer the perceptual space of the
most similar language, regardless of the acquisition order of the L1 and
L2. It is predicted that (P1) in terms of perception, participants will
identify the sounds as belonging to the most similar language,
regardless of the order of acquisition, and that (P2) in terms of
production, participants will produce vowel sounds closer to the most
similar language, again, regardless of the order of acquisition.

\section{Methods}

\subsection{Participants}

30 advanced late bilinguals participated in the study. There was a first
group of 15 L1 English, L2 Spanish participants (mean age = 20.3 years)
and a second one of 15 L1 Spanish, L2 English participants (mean age =
21.7 years). All participants had acquired their L2 in a foreign
language classroom setting (mean AoA = 19 years), and reported having
lived in a L2-speaking country for at least two years. Participants were
undergraduate students at Rutgers University at the moment of completing
the study, and received course credit after participation.

Given that previous research shows that beginner L3 learners are more
prompt to notice the acoustic contrasts between their languages
\cite{Wrembel/etal:2019}, participants were early learners of the L3.
Aside of their main two languages, all participants were enrolled in a
Introduction to Korean II course as third language. Participants were at
the end of the second semester taking the course. No participants
reported knowing another language.

\subsection{Materials}

\subsubsection{Screening Tests}

To determine eligibility for the study, participants completed a brief
online survey before conducting the experiment. To assess their
bilingual profile, participants completed an adaptation of the Language
and Social Background Questionnaire (LSBQ, \cite{Anderson/etal:2018})
and the LexTALE, (\cite{Lemhofer/Broersma:2012}).

The LSBQ \cite{Anderson/etal:2018} is a self-reporting tool that
assesses language use across various contexts, language proficiency in
multiple skills, age of acquisition, and language switching and mixing.
Its full version contains 62 items. Participants completed the
adaptation of the LSBQ in their L1. Its completion should take around 15
minutes.

The LexTALE \cite{Lemhofer/Broersma:2012} is a proficiency test
containing 60 items. Generally designed to screen bilinguals,
participants have to decide whether the word they see is a real word or
not (Yes/No decision). It is a reliable and valid test, created for
English and adapted into other languages \cite{Lemhofer/Broersma:2012}.
Participants completed the LexTALE in their L2 and Korean. The test
should take approximately 10 minutes per language to complete.

\subsubsection{Discrimination Task}

Participants completed a discrimination task so to evaluate their vowels
perception. Materials included 30 pseudo-artificial Korean words
containing the vowel sound \textipa{/E/}, 30 pseudo-artificial Korean
words containing the vowel sound \textipa{/V/}, and 30 pseudo-artificial
Korean words containing the vowel sound \textipa{/e/}. The words
containing \textipa{/E/} and \textipa{/V/} were target words, for they
included vowel sounds shared with English and Korean. The words with the
\textipa{/e/} sound were used as distractors, for that sound appears in
the three languages. Distractors were not considered in the statistical
analysis. Words were divided into two different blocks, each containing
20 words from each vowel sound. Words were recorded by a professor from
the department of Korean at Rutgers University.

\subsubsection{Reading Aloud Task}

Participants completed a reading aloud task so to evaluate their vowels
production. Materials included 20 Korean words containing the vowel
sound /u/, which were embedded into carrier phrases, such as example,
and 20 Korean words containing the vowel sound /u/, which were also
embedded into carrier phrases with the same structured. Words were
matched for length, frequency, and controlled for difficulty. Four
professors from the department of Korean at Rutgers University graded
the words' difficulty on a scale from 1 to 5. Those words which received
a score of 3 or higher were discarded and new ones were included, which
were also graded for difficulty. Sentences were divided into two
different blocks, each containing 10 sentences from each vowel sound.

\subsubsection{Procedure}

Before conducting the experiment, participants were tested for
eligibility into the study. Participants completed online the LSBQ and
LexTALE in their L2 and Korean. Once eligibility was confirmed,
participants signed the consent form and were accepted into the study.

First, they were welcomed into the RAP lab at Rutgers University. Then,
they were instructed to sit comfortably in a chair and completed the
Discrimination Task. While giving instructions, the person conducting
the experiment used both English and Spanish to favor a bilingual mode
environment. The participants were told that they would listen to Korean
monosyllables and, as soon as hearing them, would need to decide if the
words sounded more English-like or Spanish-like. Words were presented in
a Apple monitor with a white background and on black Arial 11 font.
Participants then completed three practice trials, which were not
included in the experimental trials. Then, the task started. After
finishing the two blocks, participants were invited to take a brief
rest.

After resting for a few minutes, participants completed the Reading
Aloud Task. This time, participants went to a sound-attenuated booth
where they could seat comfortably in front of a screen. They also had a
table-mounted microphone to record their speech. Participants were
instructed to read as naturally as possible each phrase. Following
\cite{Amengual:2018}, participants were asked to male a pause at the
word boundary so to reduce as much as possible the effects of the
surrounding phonological context.

\subsubsection{Acoustic Analysis}

The Korean words produced in the Reading Aloud Task were manually
segmented in Praat. Praat scripts were used to divide each participant's
recording into individual files for each target item, creating a text
grid for each token, and normalizing the peak intensity. Then, each
vowel type was added manually into the text grid. Finally, another Praat
script was used to extract formant values at the vowel midpoint for each
token and written into a .csv file.

\subsubsection{Statistic Analysis}

In the case of the Discrimination Task, a perception model was fit. The
response type was taken as a sound sensitivity measurement. Using R
\cite{RCore:2025}, responses were categorized by vowel type and then
divided between true positives (participant is able to recognize the
sound as belonging to their similar language), or false negative
(participant is not able to recognize the sound as belonging to the most
similar language). The sensitivity measure was calculated then with the
following formula \cite{Shreffler/Huecker:2023}:
\begin{equation}\label{eq:tzero}
Sensitivity = \frac{True Positives}{True Positives + False Negatives}.
\end{equation}

For the Reading Aloud Task, a production model was fit. Formant values
were compared within-subjects and identified as more L1- or L2-like.
Although proficiency was controlled, an extra model for both perception
and production was fit includingn it so to measure its effect.

\subsection{References}

\bibliographystyle{./includes/bib/IEEEtran.bst}
\bibliography{./includes/bib/icphs2023.bib}

\theendnotes

\end{document}
